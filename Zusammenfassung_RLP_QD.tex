\documentclass{article}
\usepackage{authblk}
\usepackage[utf8]{inputenc}
\usepackage[T1]{fontenc}
\usepackage[ngerman]{babel}

%============================================ Define Titlepage & packages =============================================%

\title{Naturnahe Waldwirtschaft mit der QD-Strategie - Zusammenfassung\\\vspace{1cm}}

\author{\large Andreas Hill}


\usepackage{fancyhdr}     
\usepackage{amsmath} %Paket für erweiterte math. Formeln
\usepackage[labelfont=bf]{caption}
\usepackage[font=footnotesize]{caption}
\usepackage[font=footnotesize]{subcaption}
\usepackage{graphicx}
\usepackage{caption}
\usepackage{subcaption}
\usepackage[final]{pdfpages}

\usepackage{geometry}
 \geometry{
 a4paper,
 left=25mm,
 right=25mm,
 top=30mm,
 bottom=30mm
 }

\setlength{\parindent}{0em} % Einzug bei neuen Absätzen

%------------------------------------------------------------------------------------------------%
% -------------------------------------- Main Document------------------------------------------ %

\begin{document}


%------------------------------------------------------------------------------------------------%
% -------------------------------------- Tex Settings ------------------------------------------ %

\maketitle
\thispagestyle{empty}
\newpage

\pagenumbering{arabic}
\setcounter{page}{1}

\pagestyle{fancy} %Kopfzeile und Fusszeile
\fancyfoot[C]{\thepage}
\setlength{\headsep}{15mm}


%------------------------------------------------------------------------------------------------%
% ---------------------------------- Entwicklungsphasen ------------------------------------------------ %

\section{Entwicklungsphasen}

% ---------------------- %
\subsection{Etablierung}
% ---------------------- %

\begin{itemize}

  \item frühste Waldentwicklungsphase, leitet den Generationenwechsel ein
  
  \item Aktive Steuerung / Unterstützung nur dann, wenn Entwicklungsverzögerungen oder Blockaden vorhanden sind, welche die natürliche Waldentwicklung behindert. Dann gezielt Impulse setzen oder Entwickungshemmungen lösen. Bei Klumpenbehandlung geht es um "Zieloffenheit zur Mehrwerterzeugung"
    
  \item qualitative Baummerkmale werden noch nicht oder bestenfalls ansatzweise in Erscheinung treten
  
  \item Anzahl an jungen Bäumen darauf ausgerichtet, die waldwirtschaftlich gewünschte Zahl grosskroniger Bäume von hohem Wert mit grösster Wahrscheinlichkeit zu erreichen
  
  \item Einleitung der Verjüngung nicht flächendeckend, sondern auf kleinen Teilflächen (\textit{Klumpen}) zu realisieren
  
  \item Klumpen:
  
    \begin{itemize}
      \item Einheit, auf welche bis zum Abschluss der Etablierung alle Beobachtungen und waldbaulichen Handlungen erfolgen
      \item Ziel: möglichst natürlicher, selbsttätiger ökologischer Ablauf
      \item Klumpen: auf 15\% der Fläche, 5-7 m Durchmesser im Abstand von mind. 12- max. 18 m; wenn Zielhöhen $>$ 40 m, dann auch 24 m Abstand genügend 
  	  \item für die meisten Baumarten: 15-30 junge Bäume (Buche: 40) pro Klumpen führen idR zur Ausdifferenzierung von mindestem 1 zielentsprechendem Baum (Vitalität, Wuchsform)
  	  \item Wichtig: auch randständige Bäume müssen natürliche Astreinigung erfahren können. Hierzu muss Aussenkontakt zu anderen Jungbäumen bestehen, welche mindestens gleichstark oder stärker beschatten (auchh durch Sträucher möglich wie Hasel)
  	  \item die beteiligten Baumarten müssen bezgl. ihres frühen Höhenwachstums aufeinander passen (begleitende BA darf Zielbaumart nicht überwachsen oder dominieren: Lichtbaumarten - Schattenbaumarten - Pioniere beachten)
  	  \item wenn Durchmesser (Abstand der Jungbäume) zu klein: keine genügende Astreinigung der vitalsten Bäume. Wenn Durchmesser (Abstand) zu gross: zu späte Ausdifferenzierung und Astreinigung, auch Steilastbildung möglich 
  	  \item Klumpen platzieren: ökologisch günstige Stellen auswählen, problematische Bereiche vermeiden. Klumpen im Feld markieren
    \end{itemize}
  
  \item Naturverjüngung abhängig davon, ob samenerzeugende Bäume in hinreichender Anzahl und Nähe vorhanden sind
  
  \item Verjüngungshemmnisse und Massnahmen:
  
    \begin{itemize}
       \item ungünstiger Zustand des Oberbodens (Verdichtung, Aushagerung, Streuauflage, Grasfilz, Mäuse, ...)
       \item Massnahmen immer punktwirksam und mit geringem Aufwand, d.h. auf Teilflächen in Klumpengrösse begrenzen
       \item Beachten: Wurzelbrut mancher Baumartenn wie Elsbeere, Aspe, Vogelkirsche, Kernobstarten, Linden, Feldulme, Weiden- und Erlenarten
       \item Freilegung des Mineralbodens, Entfernung der Grasnarbe, Massnahmen auf zeitpunkt der Samenverbreitung abstimmen
       \item Bucheckern, Eicheln in Boden eingraben
       \item Saat: naturnah, aber anspruchsvoll
       \item Pflanzung von Wildlingen: zum Beispiel bei Buche. Vorteil Wildlinge: Ort der Wildlingsgewinnung kann mit Zielstandort abgestimmt werden.
       \item Sämlinge aus Baumschulen als letzte Wahl (bevorzugt einjährig). Achtung: Pfahlwurzel muss intakt bleiben.
       \item Wichtig bei Pflanzung. Feinwurzeln vor Austrocknung sichern --> Zwischeneinschlag
       \item bei Pflanzung können gesellschaftsbildende Baumarten in insich artenreinen Klumpen gemischt werden (Vogelkirsche, Elsbeere, Ulme, Esche)
    \end{itemize}


   \item Lichtangebot ist Schlüssel für Etablierung. Minder schattentolerante Arten können unter stärker schattenzoleranten nicht bestehen. Beispiel: Esche unter Weissdorn. Aber: Pioniere / LichtBA können auch durch Triebbildung Schatten entkommen.
   
   \item hinsichtlich Lichtangebot immer "das grösste bestehende Risiko" bewerten (zum Beispiel aufkommende Konkurrenzvegetation)

   \item Einflussnahme auf Verjüngung nur auf Klumpen konzentrieren, auf Rest der Fläche ist natürliche Entwicklung gewünscht
   
   \item Ziel von Eingriffen ist die Etablierung zielentsprechender Baumarten zu sichern. Es geht hier nicht um die Herbeiführung eines Augenblickzustandes, sondern um die Offenhaltung von Möglichkeiten / günstigen Zuständen. Eingriffe möglichst nicht destruktiv, kann aber z.B. Knicken des Sprosses / Entfernen von Konkurrenzpflanzen einschliessen

   \item Beispiel Hemmnis durch Brombeere:
   
    \begin{itemize}
    	\item oft sehr ausgeprägt in künstlichen Wirtschaftswäldern, die von Licht- und Halblichbaumarten geprägt sind
    	\item können Verjüngung über Jahrzehnte blockieren bzw. zu Deformation von Jungpflanzen führen
    	\item Baumarten, welche mit Brombeere zurecht kommen können: Tanne , Eschen, Bergahorn --> bilden stabile Haupttriebe, welche Brombeeren ohne Verformung durchwachsen können
    	\item Baumarten, welche durch Brombeere stark deformiert werden können: Vogelkirsche, Birke, Feldahorn, Hainbuche, Stieleiche, Traubeneiche
    	\item Buche kann sich in / unter Brombeere etablieren, wird aber oft verbogen --> keine Wertholtzerzeugung möglich
    	\item Gegenmassnahmen (auf Klumpen beschränkt): 1) bodenebenes Abschneiden aller oberirdischen, 1-2 jährigen Triebe im Zeitfenster Ende Juli - Mitte August --> starke Schwächung der Brombeere; 2) komplettes ausreissen der Brombeere inkl. Wurzeln im Spätwinter 	
    \end{itemize}
   
    \item Verbissschutz-Massnahmen:
    
      \begin{itemize}
        \item einfachste \& günstigste Massnahme: wiederholter Schutz der Endknospe durch Umwickeln mit Schafswolle oder Kreppband (Methode wirkt aber nur im Winter vor Austrieb des neuen Triebes)
        \item Sommerverbiss v.a. kritisch bei gepflanzten Jungbäumen in erster Veg.periode
        \item Netzgeflechtshüllen (1-2 m hoch): kostenaufwendig
        \item Zäunung: nicht zu empfehlen (teuer, nicht immer zielführend, widerspricht naturnaher Waldwirtschaft)    
      \end{itemize}
    
    \item normale Einleitung der Etablierung: allmähliche und kleinflächig wikrsame Erhöhung des Lichtzutrittes (begünstig Schattenbaumarten). Seltener: abruptes Shaffen von Freilagen (begünstigt Lichtbaumarten, Pioniere), aber Gefahr der Etablierung von Konkurrenzvegetation wie  Gräser, Stauden usw.
    
    \item generell: Bäume erst gezielt ernten, wenn sie sich bereits verjüngt haben
    
    \item bei Ernte wichtig: Klumpen deutlich markieren und bei Fällung schonen; Schlagpflege sehr wichtig: Krone aus Klumpen entfernen oder zerkleinern, Deformationen der Jungpflanzen vermeiden durch rechtzeitiges (vor Veg.periode) Wiederaufrichten
      
\end{itemize} 
  
  
  
% ---------------------- %
\subsection{Qualifizierung}
% ---------------------- %

\begin{itemize}

 \item Beginnt, wenn sich die Jungbäume endgültig gegenüber Konkurrenz dirch Bodenvegetation, Verbiss etc. durchgesetzt haben

 \item In der Qualifizierung muss alles unterlassen werden, was den astreinigenden Seitendruck vermindert. Eine Stammzahlverminderung (Zwischenvitalisierung) fördert nur die Ausbildung von Protzen und verzögert die Kronenexpansions

 \item Minderheiten / ökologisch wertvolle Bäume werden belassen, ohne an sie spezielle Forderungen hinsichtlich Vitalität zu stellen



































\end{itemize}













%------------------------------------------------------------------------------------------------%
\end{document}







