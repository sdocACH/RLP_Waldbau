\documentclass{article}
\usepackage{authblk}
\usepackage[utf8]{inputenc}
\usepackage[T1]{fontenc}
\usepackage[ngerman]{babel}

%============================================ Define Titlepage & packages =============================================%

\title{Naturnahe Waldwirtschaft mit der QD-Strategie - Zusammenfassung\\\vspace{1cm}}

\author{\large Andreas Hill}


\usepackage{fancyhdr}     
\usepackage{amsmath} %Paket für erweiterte math. Formeln
\usepackage[labelfont=bf]{caption}
\usepackage[font=footnotesize]{caption}
\usepackage[font=footnotesize]{subcaption}
\usepackage{graphicx}
\usepackage{caption}
\usepackage{subcaption}
\usepackage[final]{pdfpages}

\usepackage{geometry}
 \geometry{
 a4paper,
 left=25mm,
 right=25mm,
 top=30mm,
 bottom=30mm
 }

\setlength{\parindent}{0em} % Einzug bei neuen Absätzen

%------------------------------------------------------------------------------------------------%
% -------------------------------------- Main Document------------------------------------------ %

\begin{document}


%------------------------------------------------------------------------------------------------%
% -------------------------------------- Tex Settings ------------------------------------------ %

\maketitle
\thispagestyle{empty}
\newpage

\pagenumbering{arabic}
\setcounter{page}{1}

\pagestyle{fancy} %Kopfzeile und Fusszeile
\fancyfoot[C]{\thepage}
\setlength{\headsep}{15mm}


%------------------------------------------------------------------------------------------------%
% ---------------------------------- Entwicklungsphasen ------------------------------------------------ %

\section{Entwicklungsphasen}

% ---------------------- %
\subsection{Etablierung}
% ---------------------- %

\begin{itemize}

  \item frühste Waldentwicklungsphase, leitet den Generationenwechsel ein
  
  \item Aktive Steuerung / Unterstützung nur dann, wenn Entwicklungsverzögerungen oder Blockaden vorhanden sind, welche die natürliche Waldentwicklung behindert. Dann gezielt Impulse setzen oder Entwickungshemmungen lösen. Bei Klumpenbehandlung geht es um "Zieloffenheit zur Mehrwerterzeugung"
    
  \item qualitative Baummerkmale werden noch nicht oder bestenfalls ansatzweise in Erscheinung treten
  
  \item Anzahl an jungen Bäumen darauf ausgerichtet, die waldwirtschaftlich gewünschte Zahl grosskroniger Bäume von hohem Wert mit grösster Wahrscheinlichkeit zu erreichen
  
  \item Einleitung der Verjüngung nicht flächendeckend, sondern auf kleinen Teilflächen (\textit{Klumpen}) zu realisieren
  
  \item Klumpen:
  
    \begin{itemize}
      \item Einheit, auf welche bis zum Abschluss der Etablierung alle Beobachtungen und waldbaulichen Handlungen erfolgen
      \item Ziel: möglichst natürlicher, selbsttätiger ökologischer Ablauf
      \item Klumpen: auf 15\% der Fläche, 5-7 m Durchmesser im Abstand von mind. 12- max. 18 m; wenn Zielhöhen $>$ 40 m, dann auch 24 m Abstand genügend 
  	  \item für die meisten Baumarten: 15-30 junge Bäume (Buche: 40) pro Klumpen führen idR zur Ausdifferenzierung von mindestem 1 zielentsprechendem Baum (Vitalität, Wuchsform)
  	  \item Wichtig: auch randständige Bäume müssen natürliche Astreinigung erfahren können. Hierzu muss Aussenkontakt zu anderen Jungbäumen bestehen, welche mindestens gleichstark oder stärker beschatten (auchh durch Sträucher möglich wie Hasel)
  	  \item die beteiligten Baumarten müssen bezgl. ihres frühen Höhenwachstums aufeinander passen (begleitende BA darf Zielbaumart nicht überwachsen oder dominieren: Lichtbaumarten - Schattenbaumarten - Pioniere beachten)
  	  \item wenn Durchmesser (Abstand der Jungbäume) zu klein: keine genügende Astreinigung der vitalsten Bäume. Wenn Durchmesser (Abstand) zu gross: zu späte Ausdifferenzierung und Astreinigung, auch Steilastbildung möglich 
  	  \item Klumpen platzieren: ökologisch günstige Stellen auswählen, problematische Bereiche vermeiden. Klumpen im Feld markieren
    \end{itemize}
  
  \item Naturverjüngung abhängig davon, ob samenerzeugende Bäume in hinreichender Anzahl und Nähe vorhanden sind
  
  \item Verjüngungshemmnisse und Massnahmen:
  
    \begin{itemize}
       \item ungünstiger Zustand des Oberbodens (Verdichtung, Aushagerung, Streuauflage, Grasfilz, Mäuse, ...)
       \item Massnahmen immer punktwirksam und mit geringem Aufwand, d.h. auf Teilflächen in Klumpengrösse begrenzen
       \item Beachten: Wurzelbrut mancher Baumartenn wie Elsbeere, Aspe, Vogelkirsche, Kernobstarten, Linden, Feldulme, Weiden- und Erlenarten
       \item Freilegung des Mineralbodens, Entfernung der Grasnarbe, Massnahmen auf zeitpunkt der Samenverbreitung abstimmen
       \item Bucheckern, Eicheln in Boden eingraben
       \item Saat: naturnah, aber anspruchsvoll
       \item Pflanzung von Wildlingen: zum Beispiel bei Buche. Vorteil Wildlinge: Ort der Wildlingsgewinnung kann mit Zielstandort abgestimmt werden.
       \item Sämlinge aus Baumschulen als letzte Wahl (bevorzugt einjährig). Achtung: Pfahlwurzel muss intakt bleiben.
       \item Wichtig bei Pflanzung. Feinwurzeln vor Austrocknung sichern --> Zwischeneinschlag
       \item bei Pflanzung können gesellschaftsbildende Baumarten in insich artenreinen Klumpen gemischt werden (Vogelkirsche, Elsbeere, Ulme, Esche)
    \end{itemize}


   \item Lichtangebot ist Schlüssel für Etablierung. Minder schattentolerante Arten können unter stärker schattenzoleranten nicht bestehen. Beispiel: Esche unter Weissdorn. Aber: Pioniere / LichtBA können auch durch Triebbildung Schatten entkommen.
   
   \item hinsichtlich Lichtangebot immer "das grösste bestehende Risiko" bewerten (zum Beispiel aufkommende Konkurrenzvegetation)

   \item Einflussnahme auf Verjüngung nur auf Klumpen konzentrieren, auf Rest der Fläche ist natürliche Entwicklung gewünscht
   
   \item Ziel von Eingriffen ist die Etablierung zielentsprechender Baumarten zu sichern. Es geht hier nicht um die Herbeiführung eines Augenblickzustandes, sondern um die Offenhaltung von Möglichkeiten / günstigen Zuständen. Eingriffe möglichst nicht destruktiv, kann aber z.B. Knicken des Sprosses / Entfernen von Konkurrenzpflanzen einschliessen

   \item Beispiel Hemmnis durch Brombeere:
   
    \begin{itemize}
    	\item oft sehr ausgeprägt in künstlichen Wirtschaftswäldern, die von Licht- und Halblichbaumarten geprägt sind
    	\item können Verjüngung über Jahrzehnte blockieren bzw. zu Deformation von Jungpflanzen führen
    	\item Baumarten, welche mit Brombeere zurecht kommen können: Tanne , Eschen, Bergahorn --> bilden stabile Haupttriebe, welche Brombeeren ohne Verformung durchwachsen können
    	\item Baumarten, welche durch Brombeere stark deformiert werden können: Vogelkirsche, Birke, Feldahorn, Hainbuche, Stieleiche, Traubeneiche
    	\item Buche kann sich in / unter Brombeere etablieren, wird aber oft verbogen --> keine Wertholtzerzeugung möglich
    	\item Gegenmassnahmen (auf Klumpen beschränkt): 1) bodenebenes Abschneiden aller oberirdischen, 1-2 jährigen Triebe im Zeitfenster Ende Juli - Mitte August --> starke Schwächung der Brombeere; 2) komplettes ausreissen der Brombeere inkl. Wurzeln im Spätwinter 	
    \end{itemize}
   
    \item Verbissschutz-Massnahmen:
    
      \begin{itemize}
        \item einfachste \& günstigste Massnahme: wiederholter Schutz der Endknospe durch Umwickeln mit Schafswolle oder Kreppband (Methode wirkt aber nur im Winter vor Austrieb des neuen Triebes)
        \item Sommerverbiss v.a. kritisch bei gepflanzten Jungbäumen in erster Veg.periode
        \item Netzgeflechtshüllen (1-2 m hoch): kostenaufwendig
        \item Zäunung: nicht zu empfehlen (teuer, nicht immer zielführend, widerspricht naturnaher Waldwirtschaft)    
      \end{itemize}
    
    \item normale Einleitung der Etablierung: allmähliche und kleinflächig wikrsame Erhöhung des Lichtzutrittes (begünstig Schattenbaumarten). Seltener: abruptes Shaffen von Freilagen (begünstigt Lichtbaumarten, Pioniere), aber Gefahr der Etablierung von Konkurrenzvegetation wie  Gräser, Stauden usw.
    
    \item generell: Bäume erst gezielt ernten, wenn sie sich bereits verjüngt haben
    
    \item bei Ernte wichtig: Klumpen deutlich markieren und bei Fällung schonen; Schlagpflege sehr wichtig: Krone aus Klumpen entfernen oder zerkleinern, Deformationen der Jungpflanzen vermeiden durch rechtzeitiges (vor Veg.periode) Wiederaufrichten
      
\end{itemize} 
  
  
  
% ---------------------- %
\subsection{Qualifizierung}
% ---------------------- %

\begin{itemize}
	
 \item Beginnt, wenn sich die Jungbäume endgültig gegenüber Konkurrenz durch Bodenvegetation, Verbiss etc. durchgesetzt haben
 
 \item Aststerben als Qualifizierungsmerkmal: Aststerben nur dann, wenn Seitenkontakt zu einer mindestens gleichstark beschattenden Baumart besteht
 
 \item Erkennen von Supervitalen, regelmässige Beurteilung der Entwicklung der Höhentriebe der Jungbäume
 
 \item Supervitale: haben im Vergleich zu Artgenossen (selbe BA, selbe Entw.phase) überlegenes Höhenwachstum 
 
 \item im Sinne der Mehrwerterzeugung sind Supervitale 'nur' Optionen
 
 \item Supervitale können auch Bäume mit unzureichender Qualität sein, welche den Raum sehr dominant besetzen können --> Protzen
 
 \item übergeformte Minderwüchsige der Vorgeneration, welche das Aufwachsen der Jungbäume verhindern --> Säue
 
 \item Erziehungswirkung (positiv) von Lichbaumarten, welche schattenertragende Optionen überwachsen haben
 
 \item innerartlicher Kontakt: überwachsene Bäume sterben auf kurz oder lang ab 
 
 \item zwischenartlicher Kontakt: Schattenerträgnis entscheidend. Jungbäume lichtbedürftiger Arten sterben bei Beschattung innerhalb weniger Jahre (z.B Waldkiefer unter Birke).
 
 \item Lichtwendigkeit und Schattentoleranz: 
	 \begin{itemize}
	 \item einseitiges Überwachsen: Bäume vieler lichtbedürftiger Arten (auch Eichen) weichen mit Hauptrieb aus (Lichtwendigkeit) --> schiefer, krummer, bogiger Wuchs (verhindert Mehrwerterzeugung)
	 
	 \item Traubeneiche: kann nicht bestehen, wenn zunächst von Birke überwachsen und anschliessend von vorbeiziehender Buche ausgedunkelt
	 
	 \item  Weisstanne / Eibe: Schattenbaumarten (noch mehr als Buche) können sehr lange im Dunkeln ausharren und auch nach 100 Jahren noch auf mit vollem Reaktionsvermögen auf Auflichtung reagieren
	 
	 \item auch Buchen, Elsbeeren und die Ahörner können nach vielen Jahren unter Eichen oder Pionieren so reagieren
	 
	 \item Schattenbaumarten sind nicht lichtwendig, sondern setzen ihr lotrechtes Höhenwachstum fort
	 
	 \item auf Freiflächen oder unter viel Lichtangebot: langsame hochwachsende schattenertragende Arten wie Buche, Hainbuche, Linde kann von Pionieren / LichtBAs (Aspe, Birke, Salweide oder Eichen) überwachsen werden. Dies vermindert nicht das Höhenwachstum der Buche, hat aber 'Erziehungswirkung' auf die Buche: -->  Verhinderung von Steilästen / waagerechte Ausrichtung der Seitenäste an der Buche / Förderung der Astreinigung.
	 
	 \item leichte Ausgleichsbewegungen kann zu leichten Stammkrümmungen bei Buche führen, wird aber durch späteres Dickenwachstum ausgeglichen
	 
	\end{itemize}
	 
 \item Ziel: Mindestanzahl an Optionen: in früher Qualifizierung ca. 5-fache (ca. 250 Optionen / ha), in späterer Qualifitierung mind. 3-fache (150 Optionen / ha) der späteren Auslesebaum-Anzahl.
 
 \item Ziel: frühzeitige Qualifizierung für möglichst grosse Kronenexpansion (Dimensionierung zum Zeitpunkt der Zuwachskulmination)
 
  \item Es geht in der Qualifizierung nicht um Förderung des Durchmesserwachstums. In der Qualifizierung muss alles unterlassen werden, was den astreinigenden Seitendruck vermindert. Eine Stammzahlverminderung (Zwischenvitalisierung) fördert nur die Ausbildung von Protzen und verzögert die Kronenexpansion

 \item Minderheiten / ökologisch wertvolle Bäume werden belassen, ohne an sie spezielle Forderungen hinsichtlich Vitalität zu stellen

 \item Brombeere stellt einen mechanischen Schutz gegen Verbiss her, Hasel und Hartriegel werden bevorzugt gefegt. Fegezeit: April-Mai
 
 \item Schälschutzmassnahmen erfordert die Festlegung auf Optionen, obwohl sich noch viel ändern kann 
 
 \item Zur Beobachtung werden Zugangslinien angelegt, wenn das Aststerben bis in 1.5 m Augenhöhe fortgeschritten ist: Breite 0.8 - max. 1 m im Abstand von ca. 20 m. Breitere Linien führen zu Innenrandeffekten. Zugangslinien bevorzugt auf alten Rückegassen anlegen.
 
 \item Beurteilung von Supervitalen und Eingriffe:
   \begin{itemize}
 
     \item Entwicklung in längstens 4-Jahres Turnus begutachten.
     
     \item Augenmerk auf Supervitale (Optionen) + Förderung von 'besonderen' Bäumen
     
     \item Prognose der weiteren Höhenentwicklung der Optionen
     
     \item für Optionen lichtbedürftiger BAs (z.B. Eiche) muss Gipfeltrieb frei bleiben (Prognose: wird er in den nächsten Jahren überwachsen ?) --> sonst droht Verbuschung des Gipfels
     
     \item Protzen entfernen, wenn sie wertvermindernde Wirkung auf Optionen(bildung) haben
     
     \item Eingriffe dann, wenn durch hohe Pionierbaumanzahlen gegenüber lichtbedüftigen Arten (z.B. Birken über Eichen)
     
     \item Eingriffe im Vorbeigehen
     
     \item typische Ausgangssituation: Hochwachsen von Pionierbaumarten (Birke, Aspe, Salweide) --> andere lichtbedürftige BAs (z.B. Kiefer, Eiche) können nicht Schritt halten --> werden in Vitalität stark eingeschränkt
    
     \item Knicken des vorwachsenden Konkurrenten: ermöglicht ersatzweise Raumbesetzung durch lichtbedürftige BAs welche gegenüber Pionieren von Natur aus erst später die Oberhand gewinnen. Evtl. müssen auch vitale Seitenäste geknickt werden, um Revitalisierung zu vermeiden
     
     \item optimaler Zeitpunkt für Knicken: Mitte Juli - Mitte August
     
     \item Knicken bei hoher Anzahl an Pionierbaumarten kann problematisch sein, weil Pioniere den Platz schnell wieder besetzen. Dann evtl. eher auf Pionierbaumart als Option setzen (Birke, Schwarzerle)
     
     \item Ringelung: Vollständige Entfernung des Bastes --> Zerfall 2 - 8 Jahre später. Schneller Eintritt der Devitalisierung, wenn zwischen April und Mai geringelt wird
     
     \item Ringelung von Pionieren auch noch kurz vor Ende des Qualifizierung einer lichtbedüftigen Option möglich
     
     \item Ringeln verhindert eine abrupte Änderung der Entwicklungsbedingungen / allmähliche Entlastung des Wettbewerbdrucks, weitere Konkurrenten werden daran gehindert, den Platz einzunehmen
     
     \item 'so lange Ringeln wie nötig, so früh Ringeln wie möglich' --> wenn Gipfeltrieb von lichtbedürftger Option schon bedeckt, dann muss geknickt werden (wegen Lichtwendigkeit)
     
     \item Ausästung: wenn Aststerben im unteren Stammteil auf lange Sicht nicht mehr erwartet werden kann --> Ästung als Notqualifizierung vor Überschreiten des kritischen Astdurchmessers (3 cm, gerade Steiläste problematisch)
     
     \item Bei Ausästung wichtig: nur Äste über kritischem Durchmesser entfernen, möglichst wenig Grünäste
     
     \item Baumentnahme: Aus Sicherheitsgründen, wenn Bäume > 12 m und BHD > 12 cm
      
     \item Problem Lianen: a) Waldgeissblatt (Strangulation bis maximal 5 m) --> im Frühjahr mit Wurzel ausreissen; b) Waldrebe (bis auf 20 m Höhe) --> Entfernen der Stöcke (nur, wenn vollständig beschattet) 
     
     
  \end{itemize}

\newpage
% ---------------------- %
\subsection{Dimensionierung}
% ---------------------- %

 \item es geht bei der Dimensionierung um das Beibehalten des hohen Durchmesserzuwachses am Ende der Qualifizierung. Es erfolgt ein Zuwachsschub durch Erlangung von Schirmfreiheit

 \item 


























































\end{itemize}




%------------------------------------------------------------------------------------------------%
% ---------------------------------- Landeswaldgesetz ------------------------------------------------ %
\newpage
\section{Landeswaldgesetz RLP}

\begin{itemize}

	\item Walddefinition:
	
		\begin{itemize}
		  \item Wald im Sinne des Gesetzes ab 0.2 Hektar und Mindesbreite von 10 m
		  \item auch kahl geschlagene, verlichtete Flächen, Waldwege, Lichtungen, Holzlagerplätze
		\end{itemize}


	\item Grundpflichten (4x): 
	
		\begin{itemize}
			
			\item \textbf{ordnungsgemäss}: Aufbau und Erhaltung gesunder und stabiler Wälder; Sicherung und Steigerung nachhaltiger Holzproduktion, unverzügliche Aufforstung unbestockter Waldflächen, standortgerechte BA-Wahl, Förderung Naturverjüngung, Walderschliessung, Bodenschutz, Verzicht auf Pflanzenschutzmittel, Vermeidung von Wildschadensverhütung durch Erzielen von entsprechenden Wilddichten. \textit{Verboten:} Kahlschläge über 0.5 ha (2 ha in Reinbeständen), vorzeitige Nutzung von Nadelbäumen ($<$ 50 Jahre) und Laubbäume ($<$ 80 Jahre) ausser Pappel, Weiden, Weichlaubhölzer
			
			\item \textbf{nachhaltig}: \textit{Nachhaltigkeit}: Dauerhafte Erhaltung des wirtschaftlichen Nutzens, der natürlichen Lebensgrundlage des Menschen, der biologischen Vielfalt und des Nutzens für die Allgemeinheit. \textit{Umweltvorsorge}: Entwicklung des Waldes hinsichtlich natürl. Lebensgrundlage für den Menschen und Nutzen für Allgemeinheit.
			
			\item \textbf{planmässig}: Pflicht für Staats-, Körperschafts- und Privatwald über 50 ha Holzbodenfläche: Aufstellung von mittelfirstigen Betriebsplänen und jährlichen Wirtschaftsplänen. Betriebspläne (Forsteinrichtung) werden durch das Land (LForsten) oder private Sachkundige aufgestellt. Kosten für Körperschaften werden in beiden Fällen vom Land übernommen. Für Privatwälder müssen die Eigentümer sich in beiden Fällen mit 25\% der Kosten beteiligen (wird also vom Land stark subventioniert). Betriebspläne müssen oberer Forstbehörde (ZdF) vorgelegt werden. Betriebsplan muss ordnungsgemäss sein und Nachhaltigkeit und Umweltvorsorge genügen.
			
			\item \textbf{sachkundig}: Befähigung für den \textit{höheren Forstdienst} erforderlich für a) FA-Leitung, b) Aufstellung Betriebsplan (Forsteinrichtung); Befähigung für den \textit{gehobenen Forstdienst} für Revierdienst erforderlich.
			
		\end{itemize}
	
	
	\item Umwandlung von Wald (Rodung/ Umwandlung in andere Bodennutzungsform, Neuanlage) nur mit Genehmigung des Forstamtes. Aktion wird gegen öffentliches Interesse abgewägt. Evtl. muss eine  Umweltverträglichkeitsprüfung durchgeführt werden. Auch im Rahmen eine Bebauungsplans prüft das Forstamt, ob Voraussetzung für eine Umwandlung besteht. 


   \item Revierdienst: 

      \begin{itemize} 

          \item Waldbesitzende müssen im Rahmen des Wirtschaftsplanes für Durchführung des Revierdienstes sorgen (obligatorisch nach fachlicher Weisung des Forstamtes in Staats- und Körperschaftswald ab 50 ha Holzbodenfläche)
          
          \item für Kleinprivatwald sollen durch die obere Forstbehörde Privatwaldbetreuungsreviere gebildet werden
          
          \item Privatwald kann durch Waldbesitzenden selbst bewirtschaftet werden, wenn ausreichende Kenntnisse für die ordnungsgemässe Bewirtschaftung gegeben sind

      \end{itemize}


   \item Staatswald: 

      \begin{itemize} 
	
	    \item soll Gemeinwohl in besonderem Masse dienen
	 
	    \item vorrangig im Staatswald sind Flächen für Biotopschutz und Naturwaldreservate auszuweisen
	
	    \item soll dem forstlichen Versuchswesen dienen
	    
	    \item wird vom Forstamt bewirtschaftet
	
      \end{itemize}


   \item Körperschaftswald: 

	 \begin{itemize} 
		
		\item soll dem Gemeinwohl dienen; hat Interesse der Gemeinde und der örtlichen Bevölkerung zu dienen, soll als wertvoller Bestandteil des Gemeindevermögens erhalten werden
		
		\item Waldbesitzende bestimmen die Ziele und die Bewirtschaftungsintensität im Rahmen der Gesetze
		
		\item Forstfachliche Leitung (Planung, Durchführung , Überwachung forstlicher Arbeiten, Nachweis der Betriebsergebnisse) wird vom Forstamt ausgeübt
		
		\item Körperschaft verfügt über Walderzeugnisse, begründet und beendigt Arbeitsverhältnisse, vergibt Aufträge an Unternehmen. Wenn Körperschaft Aufgaben selbst wahrnimmt, dann berät Forstamt. Die Aufgaben kann die Körperschaft jedoch auch dem Forstamt übertragen (das Forstamt muss in diesem Falle den Auftrag annehmen). Die Körperschaft bleibt bei Verträgen mit Dritten Vertragspartner. Die Leistungen des Forstamtes sind für die Körperschaft kostenfrei.
		
		\item Der Revierdienst kann durch staatl. Bedienstete oder Bedienstete der Körperschaft ausgeübt werden (müssen sachkundig sein). Bei staatl. Bediensteten schlägt das Forstamt Bewerber vor und die Körperschaft kann unter diesen entscheiden. Im anderen Fall ist das Forstamt anzuhören. Beim Revierdienst durch staatl. Bedienstete erstattet die Körperschaft dem Land die anteiligen Personalausgaben in Form eines Hundersatzes (Prozentsatz) der durchschnittlichen Personalausgaben (wenn unter 50 ha Holzbodenfläche, dann über Gebührensatz abgerechnet). Beim Revierdienst durch Bedienstete der Körperschaft erstattet das Land der Körperschaft die anteiligen Personalausgaben in Form eines Hundersatzes (Prozentsatz) der durchschnittlichen Personalausgaben.
		
		\item Das Forstamt stellt den Wirtschaftsplan nach den Zielsetzungen, Bedürfnissen und Wünschen der Körperschaft im Rahmen des Betriebsplanes auf. Die Körperschaft beschliesst über den Wirtschaftsplan als Bestandteil ihres Haushaltsplanes.
		
	 \end{itemize}	


   \item Privatwald: 

    \begin{itemize} 
	
	  \item Forstämter fördern den Privatwald durch Beratung. Auf Wunsch leitet das FA \textit{kostenlos} die Waldbesitzenden bei den Betriebsarbeiten an.

      \item Auf Wunsch der Waldbesitzenden wirkt das FA bei der Waldbewirtschaftung mit. Dafür sind Gebühren zu entrichten.

    \end{itemize}


   \item Forstbehörden: 

    \begin{itemize} 
	
	\item oberste Forstbehörde: Ministerium
	
	\item obere Forstbehörde: ZdF
	
	\item untere Forstbehörde: Forstamt
	
    \end{itemize}


   \item Forstaufsicht: von den Forstbehörden (staatl Forstamt) ausgeübte hoheitliche Tätigkeit; Einhaltung der Grundpflichten überprüfen, Revierdienst gewährleisten.


\end{itemize}


% ---------------------------------------------------------------------------------------------------- %
% ---------------------------------- Holzpreise ------------------------------------------------------ %
\newpage
\section{Holzpreise}

\begin{itemize} 

\item in Euro pro Festmeter ohne Rinde angegeben

\item durchschnittliche Aufarbeitungs- und Rückekosten bei Nadelholzernte ca. 20-22 EUR/Fm (Harvester und Motormanuell)

\item aktuelle Marge für Fichten-Leitsortiment ca. 65-70 EUR/Fm

\item \textbf{Nadel-Langholz}: 

    \begin{itemize} 
    	
    	\item wird wenn möglich auf 20 Meter Länge geschnitten (maximal zulässige Länge in Deutschland)
    	
    	\item Durchschnittspreise ohne Abzug von Erntekosten (Einschlag + Rücken, siehe oben) für Fichte, Douglasie und Kiefer für Stammholz-Lang der Güteklasse B/C:

		\begin{figure}[htb]
			\centering
			\resizebox{1\hsize}{!}{\includegraphics*{fig/hp_FI_Doug_Kief.png}}
		\end{figure}

   \end{itemize} 

\item \textbf{Abschnitte}:

        \begin{itemize} 
        	
        	\item für Güteklasse B/C (Mischqualität) wird 3-5 Euro/Fm unter Fichtenlangholz-Preis der Güteklasse B bezahlt (siehe Tabelle)
 
            \item Fichten-Abschnitte werden meist auf 4.10 oder 5.10 m ausgehalten und wie Stammholz einzeln in Fm erfasst 
            
        \end{itemize} 
        	

\item \textbf{Käferholz}: 1) Frisches Käferholz mit noch anhaftender Rinde: Preisabschlag von 10 EUR/Fm zu B-Holz. 2) Älteres Käferholz mit bereits abgefallener Rinde: Preisabschlag von ca. 20\%-25\% zu B-Holz.

\item \textbf{Industrieholz}-Preise 'frei Waldstrasse' (Achtung: in Raummeter gemessen):

\begin{figure}[htb]
	\centering
	\resizebox{1\hsize}{!}{\includegraphics*{fig/hp_indholz.png}}
\end{figure}        	
        	
        	
\newpage       	
\item \textbf{Langholz Buche / Eiche}:

        \begin{itemize} 
        	
           \item momentan guter markt für Buchenstammholz, Eichen-Stammholz sehr gefragt
           
           \item Holzpreise:
        	
        	\begin{figure}[htb]
        	  \centering
        	  \resizebox{1\hsize}{!}{\includegraphics*{fig/hp_Bu_Ei.png}}
            \end{figure}

        \end{itemize} 


\item Brennholzpreise Buche / Eiche (Preis für Brennholz entspricht dem Preis von ganzen Stämmen in Brennholzqualität, die ungespalten an der Waldstraße liegen ('Brennholz lang ab Waldstraße'): \textbf{Buche: 45-53 EUR/Fm}, \textbf{Eiche: 38-46 EUR/Fm}.


\item \textbf{Holzsortimente}:

	\begin{itemize}
		
		\item Je nach Länge des Stammholzes wird unterschieden zwischen a) \textbf{Baumlang (L1)}: Stammholz, das nach dem Abzopfen und ggfls. Gesundschneiden in Baumlängen ausgehalten wird und b) \textbf{Abschnitte (L2)}: Stammholz, das zu Standardlängen eingeschnitten wird.
		
		\item zusätzliche Sortierung in Güteklassen
	
	    \item Sortimente:
	
		\begin{itemize} 
			
			\item L1;B = Langholz; normale Qualität
			\item L1;C = Langholz; noch gewerblich nutzbar
			\item L2;B = Abschnitte; normale Qualität
			\item L2;C = Abschnitte; noch gewerblich nutzbar
			\item L2;Cgw = Abschnitte; geringwertig\\
			
			\item IL 1 N = Industrieholz; Langholz; normale Qualität
			\item IL 1 N/K = Industrieholz; Langholz; Mischgüte (normales und fehlerhaftes/krankes Holz)
			\item IL 2 N = Industrieholz; Abschnitte; normale Qualität
			\item IL 2 N/K =Industrieholz; Abschnitte; Mischgüte (normales und fehlerhaftes/krankes Holz)
			\item IS N = Industrieschichtholz; normale Qualität („Papierholz“, gesund, frisch, fehlerfrei)
			\item IS N/K = Industrieschichtholz; Mischgüte (normales und fehlerhaftes/krankes Holz)
	
		\end{itemize} 
	
	
	\end{itemize}






% ---------------------------------------------------------------------------------------------------- %
% ---------------------------------- Baumarten ------------------------------------------------------- %
+

\item 
  \begin{itemize}

 
  \end{itemize}






















































\end{itemize} 



%------------------------------------------------------------------------------------------------%
\end{document}







